% LaTeX resume using res.cls
\documentclass[line,margin]{res}
\usepackage{hyperref}
\usepackage{helvetica} % uses helvetica postscript font (download helvetica.sty)
%\usepackage{newcent}   % uses new century schoolbook postscript font

\begin{document}

\name{Wang, Daoyuan}
% \address used twice to have two lines of address
\address{Cellphone:\sl +86-152-1686-1267}
\address{E-mail:\sl me at daoyuan.wang}


\begin{resume}
%\section{OBJECTIVE}       A position in the field of computers with special
%                interests in applications programming, frame designing and
%                information processing.


\section{EDUCATION} {\sl Bachelor of Engineering,} Zhejiang University, Hangzhou, China \hfill Oct 2009 to Jun 2013\\
                % \sl will be bold italic in New Century Schoolbook (or
	        % any postscript font) and just slanted in
		% Computer Modern (default) font
                Enrolled in HE Zhijun Honored Class and graduated with a HE Zhijun Certification.\\
                Won {\sl Student of Excellence} in 2011-2012.
                Major: Computer Science \\
                Overall GPA $3.83/4$

\section{COMPUTER \\ SKILLS} {$\diamond$ \sl Languages \& Software:} JAVA, Scala, Python, C/C++, C\#, PHP, SQL, R, Ruby, BASIC, Assembly, JavaScript, Linux Shell, Windows Shell, Verilog HDL, Matlab, OpenCV, OpenGL, OpenMP, \LaTeX, MS Office. \\
                {$\diamond$ \sl Operating Systems:} Windows, Linux.
                
\section{WORK EXPERIENCE}
            \begin{itemize}
            \item Involved in \href{http://give.zju.edu.cn/en/portal/index.html}{GIVE Lab} during undergradute.
            \item Worked at Intel APAC R\&D (Shanghai) as a Software Engineer Intern from Sept 2012 to Feb 2013.
            \item Work for Intel SSG PRC Cloud Computing Since Jul 2013, focusing on Hadoop/Spark and related projects.
            \end{itemize}

\section{MAJOR\\ PROJECTS} 
                %{\sl Windows based time management system} \hfill Sept 2011 To May 2012 \\
                %A software that takes care of how much time the user spent working or playing when using computer and generates charts of records of selected scopes.
                 %\begin{itemize}  \itemsep -2pt % reduce space between items
                 %\item Using Windows API to collect information and using MySQL to store and index.
                 %\end{itemize}

                %{\sl QSMail} \hfill            Jun 2011 \\
                %A inner-system e-mail platform.
                % \begin{itemize}  \itemsep -2pt %reduce space between items
                % \item Cooperated with three other students.
                % \item Protocol designed by my own.
                % \end{itemize}

                %{\sl miniSQL} \hfill        Oct 2011 To Nov 2011 \\
                %A mini implement of DBMS.
                %  \begin{itemize}
                %   \item Support the functions including select, update, delete, insert, create table, drop table, create index, drop index, execfile.
                %   \item Cooperated with two other students.
                %   \end{itemize}

                 %{\sl Dormitory fees Management System} \hfill        Sept 2011 \\
                %A system developed by PHP + MySQL to management the dormitory fees.

                %{\sl C-Minus Compiler} \hfill        Jun 2012 \\
                %Course project of Compiler Design.
                %\begin{itemize}
                   %\item Generate an X86 32-bit assembly file for Nasm.
                   %\end{itemize}
                
                {\sl TAS(Transcode as A Service)} \hfill        Sept 2012 to Dec 2012\\
                Transcode as a service on Hadoop.
                  \begin{itemize}
                   \item Won 2nd place on $\textrm{Intel}^{\textregistered}$ SWPC China 2012.
                   %\item Deployed on Amazon S3.
                   \end{itemize}
                   
                {\sl Project Panthera ASE} \hfill        Jul 2013 to Feb 2014\\
                A PL/SQL like implementation and performance optimization based on HIVE.
                  \begin{itemize}
                   \item Cooperated with two other engineers, source available at \href{https://github.com/intel-hadoop/project-panthera}{Github}.
                   \item Part of $\textrm{Intel}^{\textregistered}$ Distribution for Hadoop(\sl IDH).
                   \end{itemize}

                {\sl HiBench} \hfill        March 2014 to Dec 2014\\
                HiBench is a big data benchmark suite.
                  \begin{itemize}
                   \item Open-source project, available at \href{https://github.com/intel-hadoop/HiBench}{Github}.
                   \item Released HiBench 3.0, adding MRv2 support to HiBench.
                   %\item Supported many customers all over the world.
                   \end{itemize}

                {\sl Apache Spark} \hfill        Apr 2014 to Now\\
                Apache Spark$^{\textrm{TM}}$ is a fast and general engine for large-scale data processing.
                  \begin{itemize}
                   \item Apache top-level project.
                   \item Senior contributor in the community, mainly focused on SQL module.
                   \end{itemize}
                   
                 {\sl Intel \href{https://github.com/Intel-bigdata/OAP}{OAP} (codename: Spinach)} \hfill        Feb 2017 to Now\\
                OAP is our team's efforts for ad-hoc queries support on Spark SQL.
                  \begin{itemize}
                   \item Open-source from June 2017.
                   \end{itemize}

%\section{FOREIGN LANGUAGE}
%\begin{tabular}{p{0.4\textwidth}p{0.2\textwidth}p{0.3\textwidth}}
%\hline
%{\sl CET-4} &{$572$} &\hfill Dec 2009\\
%{\sl CET-6} &{$580$} &\hfill Jun 2012\\
%{\sl TOEFL-iBT} &{$105$} &\hfill Nov 2017
%\hline
%\end{tabular}
\section{AWARDS}
            Won 3rd prize(the 19th place) in the 11th Zhejiang University Programming Contests, and 3rd prize(the 10th place) in the 12th.

\section{Translation}
            {\sl Learning Spark: Lightning-fast Data Analysis} (Chinese version) \hfill O'Reilly

\section{Talk}
            {\sl Tuning Garbage Collection for Spark Applications}\\
                \rightline {QCon Shanghai 2015, Shanghai, Oct 2015}\\
            {\sl Spinach: Run Ad-hoc Queries on top of Spark SQL}\\
                \rightline {DTCC 2017, Beijing, May 2017}\\
            {\sl OAP: Optimized Analytics Package for Spark Platform}\\
                \rightline {Spark Summit 2017, San Francisco, Jun 2017}\\
            {\sl OAP: Optimized Analytics Package for Spark Platform}\\
                \rightline {Strata Beijing 2017, Beijing, Jul 2017}\\

\section{Miscs}
            Github: \href{https://github.com/adrian-wang}{https://github.com/adrian-wang}
\end{resume}
\end{document} 
