% LaTeX resume using res.cls
\documentclass[line,margin]{res}
\usepackage{hyperref}
\usepackage{outlines}
\usepackage{helvetica} % uses helvetica postscript font (download helvetica.sty)
%\usepackage{newcent}   % uses new century schoolbook postscript font

\begin{document}

\name{Wang, Daoyuan}
% \address used twice to have two lines of address
\address{Cellphone:\sl +86-152-1686-1267}
\address{E-mail:\sl me at daoyuan.wang}


\begin{resume}

\section{EDUCATION} {\sl Bachelor of Engineering,} Zhejiang University, Hangzhou, China \hfill Oct 2009 to Jun 2013\\
                % \sl will be bold italic in New Century Schoolbook (or
	        % any postscript font) and just slanted in
		% Computer Modern (default) font
                Enrolled in HE Zhijun Honored Class and graduated with a HE Zhijun Certification.\\
                Won {\sl Student of Excellence} in 2011-2012.\\
                Involved in \href{http://give.zju.edu.cn/en/portal/index.html}{GIVE Lab} during undergradute.
                Major: Computer Science \\
                GPA $3.78/4.0$(Overall) $3.84/4.0$(Last 2 year)

\section{WORK EXPERIENCE}
    {\sl Senior Software Engineer at Intel} \hfill         Jul 2013 to Jun 2018\\
    After Daoyuan's graduation, he joined Big Data Technologies organization of Intel SSG. He works on optimization of Hadoop/Spark eco-system software, developing new features and improving performance for typical workloads.
        \begin{outline}
            \1{\sl Project Panthera ASE} \hfill        Jul 2013 to Feb 2014\\
                An approach to support PL/SQL on Apache HIVE, to provide support for legacy queries with Intel Distribution for Hadoop(\sl IDH). The project was developed by three engineers including Daoyuan, source code is available at \href{https://github.com/intel-hadoop/project-panthera}{Github}. Daoyuan redesigned the basic processing structure of co-related subquery un-nesting, improving the passing rate of queries a lot. He committed a lot of code to this project involving bug fixing and new features such as error tracing. It was part of Intel Distribution for Hadoop(\sl IDH).

             \1{\sl HiBench} \hfill        March 2014 to Dec 2014\\
                Originated in Big Data Technologies, HiBench is a widely adopted big data benchmark suite in the industry. Before Daoyuan's contributions, HiBench could only work on MapReduce v1 of Apache Hadoop. Daoyuan fixed a lot of compatibility issues for different Hadoop versions, and finally made it support MapReduce v2 of Apache Hadoop (Hadoop Yarn), with the official release of HiBench 3.0. He then maintained this project till the end of 2014, answering question from community. This project is open-source, available at \href{https://github.com/intel-hadoop/HiBench}{Github}.
                   
            \1{\sl Apache Spark} \hfill        Apr 2014 to Sept 2016 \\
                Apache Spark is a fast and general engine for large-scale data processing. Apache Spark has become a widely used platform for computing in both clusters and clouds. Daoyuan contributed a lot of code to Apache Spark, mainly focused on Spark SQL, including new features, bug fixes and performance optimization for Intel Architecture.
                   
             \1{\sl Intel \href{https://github.com/Intel-bigdata/OAP}{OAP} (codename: Spinach)} \hfill        Feb 2017 to Now\\
                OAP is the efforts from Intel BDT for ad-hoc queries support on Spark SQL. A lot of companies have adopted Apache Spark as their default analysis engine for large volume of data, and data scientist may require the analysis platform as a real-time query engine. While Spark is not designed for ad-hoc queries, OAP is the spark package to unlock the power of hardware, accelerating query execution using mechanism like indexing and caching, as well as unlock the power of emerging hardware devices. \\
                Baidu, one of the largest internet companies in the world, has been using OAP in their real-time analysis engine in production for advertisement strategies. And they reported a 1.5x-5x performance gain using OAP.\\
                This project is open-sourced from Jun 2017. Daoyuan is a lead developer of this project.
        \end{outline}

      {\sl Software Engineer Intern at Intel} \hfill         Sept 2012 to Dec 2012\\
             Daoyuan joined Intel IT in 2012 as software engineer intern. He worked on the project of TAS(Transcode as a service) on Hadoop, providing transcoding service using Hadoop platform and Intel hardware transcoding technologies. He successfully traced down a BSOD bug, which was later identified as a bug in the driver program of Intel Graphics. He also developed a thorough test framework for the project, to enable automatic regression tests.
                  \begin{itemize}
                   \item Won 2nd place on $\textrm{Intel}^{\textregistered}$ SWPC China 2012.
                   \end{itemize}

\section{COMPUTER \\ SKILLS} {$\diamond$ \sl Languages \& Software:} Spark, Hive, Hadoop, Scala, JAVA, Python, SQL, C/C++, C\#, PHP, R, Ruby, BASIC, Assembly, JavaScript, Linux Shell, Windows Shell, Verilog HDL, Matlab, OpenCV, OpenGL, \LaTeX, MS Office. \\

%\section{STANDARD TESTS}
%\begin{tabular}{p{0.4\textwidth}p{0.2\textwidth}p{0.3\textwidth}}
%\hline
%{\sl TOEFL-iBT} &{$105$} &\hfill Nov 2017\\
%{\sl GRE} &{$152+170+3$} &\hfill Nov 2017\\
%\hline
%\end{tabular}

\section{AWARDS}
    \begin{itemize}
            \item Third prize(the 19th place) in the 11th Zhejiang University Programming Contests
            \item Third prize(the 10th place) in the 12th Zhejiang University Programming Contests
            \item Intel SSG 2017 Q3 Group Recognition Award
    \end{itemize}

\section{TRANSLATION}
            {\sl Learning Spark: Lightning-fast Data Analysis} (Chinese version) \hfill O'Reilly

\section{PUBLIC TALKS}
            {\sl Tuning Garbage Collection for Spark Applications}\\
                \rightline {QCon Shanghai 2015, Shanghai, Oct 2015}\\
            {\sl Spinach: Run Ad-hoc Queries on top of Spark SQL}\\
                \rightline {DTCC 2017, Beijing, May 2017}\\
            {\sl OAP: Optimized Analytics Package for Spark Platform}\\
                \rightline {Spark Summit 2017, San Francisco, Jun 2017}\\
            {\sl OAP: Optimized Analytics Package for Spark Platform}\\
                \rightline {Strata Beijing 2017, Beijing, Jul 2017}\\

\section{MISCS}
            \begin{tabular}{p{0.3\textwidth}p{0.7\textwidth}}
            {Github} &\href{https://github.com/adrian-wang}{https://github.com/adrian-wang}\\
            {Homepage} &\href{http://daoyuan.wang}{http://daoyuan.wang}\\
            \end{tabular}
\end{resume}
\end{document} 
