% LaTeX resume using res.cls
%\XeTeXinputencoding "GB2312"
\documentclass[line,margin]{res}
\usepackage{xeCJK}
\setCJKmainfont{STHeiti}
\usepackage{hyperref}
%\usepackage{helvetica} % uses helvetica postscript font (download helvetica.sty)
%\usepackage{newcent}   % uses new century schoolbook postscript font

\begin{document}

\name{王道远}
% \address used twice to have two lines of address
\address{联系电话:\sl +86-152-1686-1267}
\address{E-mail:\sl me at daoyuan.wang}


\begin{resume}

\section{教育经历} {\sl 何志均班} 计算机科学与技术, 浙江大学, 本科 \hfill 2009/08 至 2013/06\\
                总GPA $86/100$\\
                获浙江大学2011-2012学年三好学生\\
                获学士学位,何志均荣誉证书\\

\section{工作经历} 
    {\sl 资深软件工程师@英特尔亚太研发有限公司} \hfill         2013年七月至今\\
    毕业后加入英特尔软件与服务事业部(SSG)大数据部门,从事Hadoop/Spark大数据生态系统软件优化工作,结合实际用例为平台开发新功能并进行性能调优。\\
        \begin{itemize}
            \item{\sl Panthera ASE} \hfill        2013年7月至2014年2月\\
                该项目尝试基于Apache Hive支持PL/SQL语法,并被集成到IDH(Intel的Hadoop发行版)中,支持客户一些固有查询无需修改即可迁移到Hive平台。项目主要共由三人进行开发,代码托管在\href{https://github.com/intel-hadoop/project-panthera}{Github}上。负责重新设计了关联子查询的处理逻辑,大大提高了对复杂查询语句的支持。另外,还修复了很多错误,开发了很多新功能,提高了查询支持度,还实现了一套针对错误查询语句报错的机制。\\

             \item{\sl HiBench} \hfill        2014年3月至2014年12月\\
                HiBench是由英特尔大数据部门发起的开源大数据基准测试工具,在业界有比较广泛的应用。在本人参与项目前,HiBench仅支持Hadoop 0.x和1.x的提供的MRv1。本人修复了许多关于不同Hadoop版本的兼容性问题,并确保HiBench 3.0可以同时支持MRv1和MRv2,并负责发布了HiBench 3.0版本。该项目托管于\href{https://github.com/intel-hadoop/HiBench}{Github}。\\
                   
            {\sl Apache Spark} \hfill        2014年4月至2016年9月 \\
                Apache Spark是一个快速的分布式通用大规模数据计算平台。Apache Spark在集群和公有云/私有云中都有广泛的部署和应用。本人贡献了很多代码到Spark开源社区,主要集中在Spark SQL模块。结合用户的需求与反馈以及对程序的测试分析,所贡献的代码包括新功能实现、错误修复、性能优化等各方面,提高了Spark的易用性与功能,优化了性能。期间翻译出版了《Spark快速大数据分析一书》。\\
                   
             {\sl Intel \href{https://github.com/Intel-bigdata/OAP}{OAP} (曾用名Spinach)} \hfill        2017年2月至今\\
                OAP是英特尔大数据团队基于Spark SQL实现的即席查询加速库。许多公司在生产环境中尝试了部署Spark作为默认的大规模数据分析引擎,让数据科学家通过平台使用SQL进行一些数据分析,然而原生的Spark SQL性能可能无法满足即席查询的需求,毕竟Spark并不是为这种场景专门优化的。OAP是我们提供的Spark SQL优化包,充分利用集群的硬件条件,使用索引与缓存等机制加速查询的执行,同时对新硬件提供支持。\\
                该项目于2017年6月开源,本人为项目核心开发者,主导了0.2与0.3版本的开发工作。百度在生产环境的广告策略实时分析引擎中使用了OAP包,并在真实的查询中观察到了1.5倍至5倍的性能提升。\\

        \end{itemize}

      {\sl 实习软件工程师@英特尔亚太研发有限公司} \hfill         2012年9月至2012年12月\\
             以实习生身份工作于英特尔IT FLEX部门,参与基于Hadoop实现的TAS(Transcode as a service)项目。该项目使用Hadoop平台和英特尔集成显卡硬件解码功能提供视频转码服务。在测试中发现了系统隔一段时间会随机蓝屏的错误,后确定为显卡驱动的错误。负责使用C\#开发了项目的自动化测试框架,进行日常回归测试。

\section{计算机技能} {$\diamond$ \sl 编程语言与软件:} Spark, Hive, Hadoop, Scala, JAVA, Python, SQL, C/C++, C\#, PHP, R, Ruby, BASIC, Assembly, JavaScript, Linux Shell, Windows Shell, Matlab, OpenCV, OpenGL, \LaTeX, MS Office. \\

%\section{外语考试}
%\begin{tabular}{p{0.4\textwidth}p{0.2\textwidth}p{0.3\textwidth}}
%\hline
%{\sl CET-4} &{$572$} &{\hfill 2009/12}\\
%{\sl CET-6} &{$580$} &{\hfill 2012/06}\\
%{\sl TOEFL-iBT} &{$105$} &{\hfill 2017/11}
%\hline
%\end{tabular}
\section{获奖}
            第11届浙江大学大学生程序设计竞赛三等奖(总第19名).\\
            第12届浙江大学大学生程序设计竞赛三等奖(总第10名).

\section{翻译著作}
            {\sl Spark快速大数据分析} (Learning Spark: Lightning-fast Data Analysis中文版) \\
            \hfill 图灵程序设计丛书,人民邮电出版社

\section{演讲}
            {\sl Spark应用GC调优}\\
                \rightline {QCon上海 2015, 2015年10月}\\
            {\sl Spinach: 使用Spark SQL执行Ad-hoc查询}\\
                \rightline {中国数据库技术大会 2017,2017年5月,北京}\\
            {\sl OAP: Optimized Analytics Package for Spark Platform}\\
                \rightline {Spark Summit 2017,2017年6月,旧金山}\\
            {\sl OAP: Optimized Analytics Package for Spark Platform}\\
                \rightline {Strata Beijing 2017,2017年7月,北京}\\

\section{相关链接}
            \begin{tabular}{p{0.3\textwidth}p{0.7\textwidth}}
            {Github} &\href{https://github.com/adrian-wang}{https://github.com/adrian-wang}\\
            {个人主页} &\href{http://daoyuan.wang}{http://daoyuan.wang}\\
            \end{tabular}
\end{resume}
\end{document}







